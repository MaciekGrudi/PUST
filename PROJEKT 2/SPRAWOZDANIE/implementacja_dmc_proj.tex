\chapter{Implementacja algorytmu DMC}

W języku Matlab zaimplemetnowano program do symulacji działania algorytmu DMC, pozwalającym na prace regulatora w trybie z uwzględnieniem zakłócenia mierzalnego, jak i bez jego uwzględniania.

Algorytm DMC zaimplemetnowano w wersji oszczędnej. Jako prawo sterowania stosowano wzór \ref{prawo_ster_nz} w przypadku, gdy nie uwzględniano zakłócenia mierzalnego.

\begin{equation}
\Delta u\left(k\right)={K}_{e}e\left(k\right)-{\stackrel{-}{K}}_{1}{M}^{p}\Delta {u}^{p}(k)
\label{prawo_ster_nz}
\end{equation}

W przypadku uwzględniania zakłocenia mierzalnego jako prawo sterowania stosowano wzór \ref{prawo_ster_z}

\begin{equation}
\Delta u(k)={K}_{e}e(k)-{\stackrel{-}{K}}_{1}{M}^{p}\Delta{u}^{p}(k)-{\stackrel{-}{K}}_{1}{M}^{p}_{z}\Delta{z}^{p}(k)
\label{prawo_ster_z}
\end{equation}

Parametry {$K_e$} , {${{\stackrel{-}{K}}_{1}}$} , {${M}^{p}$} oraz {${M}^{p}_{z}$} są stałe dla każdego przebiegu i wyznaczane na podstawie uzyskanych odpowiedzi skokowych $\boldsymbol{s}$ , $\boldsymbol{s_z}$ i zadanych parametrów $D$ , $N$ , $N_u$ , $\lambda$. Uzyskiwanie parametrów prezentuje poniższy listing:

\begin{lstlisting}[style=Matlab-editor]
M = zeros(N,Nu);
M(:,1) = s(1:N);

for i = 2:Nu
M(i:N,i)=M(1:N-i+1,1);
end;

MP = zeros(N,D-1);
for i = 1:D-1
MP(:,i) = s((i+1):(i+N))-s(i);
end;

MPz = zeros(N,Dz);
for i = 1:Dz
MPz(:,i) = sz((i+1):(i+N))-sz(i);
end;

K = (M' * M + lambda * eye(Nu))\M';
K1 = K(1,:);
Ku = K1 * MP;
Ke = sum(K1);

DUP = zeros(D-1,1);
DZP = zeros(Dz,1);

Y(1:t)=Ypp;
U(1:t)=Upp;

y(1:t) = Y(1:t) - Ypp;
u(1:t) = U(1:t) - Upp;
\end{lstlisting}

Dla wyznaczonych parametrów sterowania przeprowadzano symulację procesu dla różnych przegiegów zakłocenia $Z(k)$ oraz stałej trajektorii wartości zadanej $Y^{zad}(k)$ w postaci skoku z wartości 0 do wartości 1 w chwili odpowiadającej 21 próbce sygnału.

\begin{lstlisting}[style=Matlab-editor]
for k=t:kk

Y(k) = symulacja_obiektu7y(U(k-4),U(k-5),Z(k-1),Z(k-2),Y(k-1),Y(k-2));
y(k) = Y(k) - Ypp;

DUP = [u(k-1) - u(k-2) ; DUP(1:D-2)];
DZP = [z(k-1) - z(k-2) ; DZP(1:Dz-1)];

e = yzad(k) - y(k);

du = Ke*e - Ku*DUP;

if distortion ~= 0
du = Ke*e - K1*MP*DUP - K1*MPz*DZP;
end;

u(k) = du(1)+u(k-1);
U(k) = u(k)+Upp;
%U(k)=1;

end;

e = (yzad-y)*(yzad-y)';
E = (Yzad-Y)*(Yzad-Y)';
\end{lstlisting}

