\chapter{Wstęp}

Celem projektu była implementacja, weryfikacja poprawności działania i~dobór parametrów algorytmów regulacji jednowymiarowego procesu laboratoryjnego. Tym razem jednak (w przeciwieństwie do projektu 1) braliśmy pod uwagę również zakłócenie występujące w układzie. W tym przypadku jest to zakłócenie mierzalne, a jego wartością możemy sterować za pomocą programu. Należy pamiętać, że w prawdziwych układach regulacji często nie da się go w tak łatwy sposób zmierzyć, a tym bardziej nim sterować, jednak w ramach niniejszego laboratorium możemy sobie na to pozwolić.

W pierwszej części projektu zajmowaliśmy się przygotowanym przez prowadzącego obiektem symulowanym. W MATLABie napisaliśmy programy analizujące odpowiedzi skokowe obiektu i implementujące algorytm regulacji DMC. W drugiej części używaliśmy sprzętu w laboratorium i wykonywaliśmy działania podobne do tych, którymi traktowaliśmy obiekt symulowany.