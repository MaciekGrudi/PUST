\chapter{Wnioski}

Przeprowadzone eksperymenty dowodzą, że w rozważanym typie procesu jednowymiarowego z opóźnieniem oba algorytmy radzą sobie ze sterowaniem - nie pojawiają się znaczące przeregulowania ani rosnące oscylacje. Niemniej jednak, zgodnie z hipotezą postawioną na początku, algorytm DMC przeprowadza regulację o wiele sprawniej i dokładniej dzięki predykcjom przyszłego sterowania wykonywanym na każdym kroku i to właśnie tego algorytmu powinniśmy używać w przypadku projektowania regulatora dla obiektów podobnych do testowanego przez nas stanowiska grzejąco-chłodzącego.

Warto zwrócić uwagę na fakt, że obiekt zarówno symulowany, jak i rzeczywisty był bardzo uproszczony - pomijaliśmy wpływ jakichkolwiek zakłóceń na jego działanie i tak naprawdę nie miało się co tam popsuć. Aby lepiej odwzorować procesy realizowane w prawdziwym świecie (np. przemyśle), należałoby jeszcze zbadać zachowanie układu w warunkach trudniejszych (z różnego rodzaju zakłóceniami) i dopiero na tej podstawie dobrać. Takie eksperymenty i rozważania przeprowadzać będziemy na kolejnych zadaniach laboratoryjnych.