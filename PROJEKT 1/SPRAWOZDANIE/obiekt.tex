\chapter{Sterowanie rzeczywistym obiektem - część laboratoryjna}
Zadanie laboratoryjne polegało na zebraniu danych z rzeczywistego obiektu oraz implementacji na nim algorytmów regulacji PID i DMC. Badanym obiektem było stanowisko grzewczo-chłodzące. Obiekt ten charakteryzuje się opóźnieniem na wejściu oraz wolną dynamiką. Taka charakterystyka obiektu wpływa na efektywność regulat 
\section{Odpowiedzi skokowe}

\section{Programy do symulacji algorytmów}

\section{Regulator PID}

\section{Algorytm DMC}